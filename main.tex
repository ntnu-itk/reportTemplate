\documentclass[11pt, a4paper, USenglish]{article} % change ``USenglish'' to ``norsk'' if applicable.

\usepackage{packages} % Contains all included packages. See packages.sty.
\addbibresource{bibliography.bib} % Makes the bibliography file available to biblatex.
\makeglossaries

\begin{document}
%% EIT
\newacronym{eit}{EIT}{Experts in Teamwork}
\newacronym{api}{API}{Application Programing Interface}
\newacronym{imu}{IMU}{Inertial Measurement Unit}
\newacronym{uwp}{UWP}{Universal Windows Platform}
\newacronym{ide}{IDE}{Integrated Development Environment}
\newacronym{mqtt}{MQTT}{Message Queueing Telemetry Transport}
\newacronym{os}{OS}{Operating System}
\newacronym{http}{HTTP}{HyperText Transfer Protocol}
\newacronym{json}{JSON}{JavaScript Object Notation}
\newacronym{gps}{GPS}{Global Positioning System}

%% Estimering
\newacronym{blue}{BLUE}{Best Linear Unbiased Estimator}
\newacronym{snr}{SNR}{Signal-to-Noise}
\newacronym{ppm}{ppm}{parts per million}
\newacronym{lo}{LO}{Local Oscillator}
\newacronym{crlb}{CRLB}{Cramer-Rao Lower Bound}
\newacronym{mle}{MLE}{Maximum Likelihood Estimator}
\newacronym{iid}{iid}{independent and identically distributed}
% Titlepage
\title{LaTeX Lab Report Template}
\author{Group 00\\Student 70000\\Student 70001\\Student 70002}
\date{\today}
\begin{titlepage}
    \maketitle
    \begin{figure}
    \centering
    \includegraphics[width=0.5\textwidth]{figures/logo/itk_ntnu.jpg}\\
    Department of Engineering Cybernetics
    \end{figure}
    \thispagestyle{empty}
\end{titlepage}
\pagenumbering{Roman}
\setlength\parindent{0pt}

% Abstract
\newpage
\input{abstract}


% TOC
\newpage
\tableofcontents


\newpage
\section*{Figures}
\addcontentsline{toc}{section}{\numberline{}List of figures}
\renewcommand*\listfigurename{}
\listoffigures
\newpage
\section*{Tables}
\addcontentsline{toc}{section}{\numberline{}List of tables}
\renewcommand*\listtablename{}
\listoftables
%eksempel på tabell
\renewcommand\tablename{Table} %bestemmer navn

% Main content

\newpage
\pagenumbering{arabic}
\setcounter{page}{1}
\include{intro}
\include{part1}
\include{part2}
\include{part3}
\include{conclusion}
\include{appendix}
% \input simply inserts the contents of the file, while \include forces a \newpage.
% See \input vs. \include: http://tex.stackexchange.com/questions/246/when-should-i-use-input-vs-include

% https://tex.stackexchange.com/questions/86666/how-to-create-both-list-of-abbreviations-and-list-of-nomenclature-using-nomencl
\newpage
\addcontentsline{toc}{section}{Abbreviations}
\printglossary[type=\acronymtype,title=Abbreviations]
\printglossary[title=Nomenclature]

% References
\newpage
\addcontentsline{toc}{section}{References}
\printbibliography{}
\label{sec:bibliography}

\end{document}
